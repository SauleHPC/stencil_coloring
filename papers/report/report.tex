\documentclass{article}

\usepackage{fullpage}

\usepackage{color}
\usepackage{soul}
\newcommand{\highlightsol}[1]{{\color{blue}#1}}


\begin{document}

\section{Problem definition}

Graphs. $G=(V,E)$

We are looking at stencil graphs. 2D stencil graphs with different
neighborhood.  The vertices are numbered with a coordinate $(i,j)$. So
a $N \times M$ stencil has vertices for $(0,0), (0,1), \dots (0, N-1),
(1,0), (1,1), \dots (M-1, N-1)$.  We are considering three different
neighborhood.

\paragraph{5-point stencil} where vertex $(i,j)$ is connected to $(i-1,j)$,
$(i+1,j)$, $(i,j-1)$, $(i,j+1)$

\paragraph{9-point box stencil} where vertex $(i,j)$ is connected to
$(i-1,j-1)$, $(i-1,j)$, $(i-1,j+1)$,
$(i,j-1)$, $(i,j+1)$,
$(i+1,j-1)$, $(i+1,j)$, $(i+1,j+1)$.

\paragraph{9-point star stencil} where vertex $(i,j)$ is connected to
$(i-2,j)$, $(i-1,j)$, $(i,j-1)$, $(i,j-2)$, $(i,j+1)$, $(i,j+2)$,
$(i+1,j)$, $(i+2,j)$.

All of these neighborhoods are assuming the neighbors exist (are
within the $N \times M$ stencil).


The problem we are looking at is \textbf{star coloring}. A star coloring of a
graph is a function on the vertices $C: V \rightarrow N$ such that,
$\forall (u,v) \in E, C(u) \neq C(v)$ and such that every chain of 4
distinct vertices in the graph has at least 3 colors.

An equivalent (?) definition of a star coloring is that the graph
induced by the ertices of any two colors is a forest of star graphs.

We want to find the star coloring that uses the fewer number of colors.

\section{ILP}

Wrote the ILP to solve the problem optimally. Here is what we know
from the ILP.

On 5-pt stencil. The optimal star coloring of a $3 \times 3$ stencil
has 4 colors, while $4 \times 4$ needs 5 colors. YOu can easily find a
5 color solution for 5-pt stencil. For instance, you can color a 5pt stencil
centered on a vertex with 5 different colors and then tile the plane
with that pattern to make a valid 5 color star coloring.

On 9-pt box stencil. The $3 \times 3$ stencil needs 5 colors, the $4
\times 4$ stencil needs 7 colors, and from sizes of $5 \times 5$,
there are 8-color solutions up to stencils of $48 \times 48$.

On 9-pt star stencil. The $3 \times 3$ stencil and $4 \times 4$
stencil need 6 colors. The $5 \times 5$ stencil and $6 \times 6$
stencil need 7 colors. 
\section{On tooling}

To look at these coloring problems, it is helpful to have good
tooling. A common problem is that if you admit a coloring $C$, then
any permutation of that coloring $C$ also works. The transpose of the
coloring also works $C'(i,j) = C(j,i)$ on square stencils. And the
horizontal and vertical mirroring also work.

A useful idea is to normalize the coloring using some kind of rule to
make the coloring look more similar to leveraging that. A simple idea
is to permute the coloring to have 0 at $(0,0)$, then use colors the
smallest colors for the first box of $2\time 2$, then the smallest
colors for the first box of $3 \times 3$, etc.

This does not solve all issues, but makes the solutions much easier to
look at.

\section{All 9pt-box stencils are 8 colorable}

There are ``cyclic'' solution to star coloring 9pt-box
stencils. Forcing a cyclic solution of 1D-period $4$ seem to always
lead to a 8 coloring (that is to say $C(i,j) = C(i-4,j)$.) They have
been found up to stencils of size $48 \times 48$.

And there is a valid 2D-cyclic coloring of period $(4,30)$ with 8
colors. Here is a solution.

~\\
\highlightsol{014260142601426014260142601426}014260142601426014\\
\highlightsol{230547631527034567132507436517}230547631527034567\\
\highlightsol{142601426014260142601426014260}142601426014260142\\
\highlightsol{567132507436517230547631527034}567132507436517230\\
014260142601426014260142601426014260142601426014\\
230547631527034567132507436517230547631527034567\\
142601426014260142601426014260142601426014260142\\
567132507436517230547631527034567132507436517230\\
014260142601426014260142601426014260142601426014\\
230547631527034567132507436517230547631527034567\\

Some notes on that solution. Note that the first row is a 5 color pattern that repeats:\\
\highlightsol{01426}0142601426014260142601426014260142601426014\\

Note that the third row is the same as the first row but phase shifted by 1:\\
1426\highlightsol{01426}014260142601426014260142601426014260142\\

The second and fourth row are also phase shifted by a much larger amount:\\
014260142601426014260142601426\\
\highlightsol{230547631527034}567132507436517\\
142601426014260142601426014260\\
567132507436517\highlightsol{230547631527034}\\

Some questions:
\begin{itemize}
\item Is this the smallest period? Do we care about finding a smaller period?
\item Does this pattern extend 3D box stencil? Do we care about that?
\item Can this help us find pattern for 9pt-star stencils?
\end{itemize}


\section{On 9pt-star star coloring}

For 9pt star, still don't have a clean solution.


\end{document}
